% !TeX root = ./main.tex

% --------------------------------------------------
% 資訊設定(Information Configs)
% --------------------------------------------------

\ntusetup{
  university*   = {National Taiwan University},
  university    = {國立臺灣大學},
  college       = {電機資訊學院},
  college*      = {College of Electrical Engineering and Computer Science},
  institute     = {光電工程學研究所},
  institute*    = {Graduate Institute of Photonics and Optoelectronics},
  title         = {基於矽光平台上之 O-band 1$\times$4 分波(解)多工濾波器},
  title*        = {O-band 1$\times$4 wavelength division (de)multiplexing filters on Silicon Photonics}, 
  author        = {鍾國方},
  author*       = {Kuo-Fang Chung},
  ID            = {D08941008},
  advisor       = {黃定洧},
  advisor*      = {Ding-Wei Huang},
  date          = {2022-09-26},         % 若註解掉,則預設為當天
  oral-date     = {2022-09-26},         % 若註解掉,則預設為當天
  DOI           = {10.5566/NTU2022XXXXX},
  keywords      = {矽光子, 陣列波導光柵, 波導布拉格光柵, 多模波導, 分波(解)多工器},
  keywords*     = {Silicon Photonics, waveguide Bragg grating, multi-mode waveguide, wavelength division (de)multiplexer},
}

% --------------------------------------------------
% 加載套件(Include Packages)
% --------------------------------------------------

% \usepackage[sort&compress]{natbib}      % 參考文獻
\usepackage[noadjust]{cite}     % replace with natbib

\usepackage{amsmath, amsthm, amssymb}   % 數學環境
\usepackage{ulem, CJKulem}              % 下劃線、雙下劃線與波浪紋效果
\usepackage{booktabs}                   % 改善表格設置
\usepackage{multirow}                   % 合併儲存格
\usepackage{diagbox}                    % 插入表格反斜線
\usepackage{array}                      % 調整表格高度
\usepackage{longtable}                  % 支援跨頁長表格
\usepackage{paralist}                   % 列表環境
\usepackage{lipsum}                     % 英文亂字
\usepackage{zhlipsum}                   % 中文亂字
\usepackage{xstring}        % for IEEEtranDOI.bst
%............... for narrow-width Delta symbol in equation and Times new roman -like font...............%
\usepackage{mathtools}
% \usepackage{libertine}
\usepackage[slantedGreek]{newtxmath}        % for \upDelta symbol
%....................................................................................................................................................

\newcolumntype{P}[1]{>{\centering\arraybackslash\small}p{#1}}
% \newcolumntype{M}[1]{>{\centering\arraybackslash}m{#1}}
\newcolumntype{M}[1]{>{\centering\arraybackslash\small}m{#1}}
%............... makecell for change line within the cell .......................
\usepackage{makecell}
%......................................................................................................

%............... for caption fontsize ................................
%\usepackage[format=hang,font={small,bf}]{caption}
%\usepackage[format=hang,font=small,labelfont=bf]{caption}
% \usepackage[format=hang,font=small,labelfont={large,bf}]{caption}
% \usepackage[format=plain,font=small,labelfont={large,bf}]{caption}
\usepackage[format=plain,font=small,labelfont={normalsize,bf}]{caption}
%...............................................................................

% \usepackage[utf8x]{inputenc}        % for unicode

\newcounter{subeq}
\newcommand{\stags}{
\addtocounter{equation}{+1}
\setcounter{subeq}{0}}
\newcommand{\stag}{%
    \addtocounter{subeq}{1}%
    \theequation\alph{subeq}%
                    }


% --------------------------------------------------
% 套件設定(Packages Settings)
% --------------------------------------------------
% \colorlet{ured}{red!60!gray}
% \colorlet{ured}{red!60!blue}
% \def\ur#1{{\color{ured}#1}}
\def\ur#1{#1}
